
\begin{center}
    \small
        Farmacêutico e Mestre em Ciências Farmacêuticas, em meu projeto de mestrado trabalhei extensivamente com técnicas
        de reologia para a caracterização de produtos e matérias-primas para cosméticos.
        Minha experiência acadêmica alavancou meu aprendizado em ferramentas como R e python para otimizar planejamento
        de experimentos e \emph{storytelling} através da visualização gráfica, relatórios automatizados e \emph{dashboards}.
        A experiência na indústria me levou ao desenvolvimento de \emph{soft skills}, liderança e planejamento de projetos.
\end{center}
    
\section{Experiência}

% 1 - Data final
% 2 - Data inicial
% 3 - Cargo
% 4 - Empresa / Instituição
% 5 - Descrição
% 6 - Local


\entry{atual}{mar/2021}
{Analista de Garantia da Qualidade}
{Cosmetici}
{
    Elaboraçao e revisão de documentos do SGQ;
    suporte e treinamento aos setores de produção;
    tratamento de desvios de qualidade em diversos setores;
    implementação de ferramentas e sistemas de melhoria de \emph{workflow}
    dos processos de fabricação.
}
{Faz. Rio Grande - PR}



% \entry{atual}{mar/2021}
% {Analista da Qualidade}
% {Cosmetici}
% {
%     Acompanhamento dos processos de fabricação;
%     implementação e validação de metodologias de análise;
%     tratamento de desvios de qualidade em diversos setores;
%     implementação de ferramentas e sistemas de melhoria de \emph{workflow} 
%     dentro do setor de qualidade.
% }
% {Faz. Rio Grande - PR}

% \entry{atual}{mar/2021}
% {Analista da Qualidade}
% {Cosmetici}
% {
%     Análise físico-química de matérias-primas e produtos,
%     implementação e validação de metodologias de análise;
%     tratamento de desvios de qualidade em diversos setores;
%     monitoramento de índices de performance e 
%     liberação de insumos para fabricação e expedição.
% }
%     {Faz. Rio Grande - PR}



% \entry{atual}{mar/2021}
% {Coordenador de Qualidade}
% {Cosmetici}
% {
% Monitoramento de processo de fabricação e envase.
% Acompanhamento de não-conformidades e implementação de
% sistema de melhoria contínua e gerenciamento de riscos.
% Atuo na atualização de procedimentos e protocolos de validação dos setores
% de controle de qualidade e de fabricação.
% }
% {Faz. Rio Grande - PR}

% \entry{atual}{ago/2021}
% {Supervisor Assuntos Regulatórios}
% {Cosmetici}
% {
% Responsabilidades de Assuntos Regulatórios
% }
% {Faz. Rio Grande - PR}

% \entry{ago/2021}{mar/2021}
% {Analista de Garantia da Qualidade}
% {Atsum / Cosmetici}
% {
% Análise físico-química de matérias-primas e materiais de embalagem.
% Acompanhamento de processos de fabricação, envase e controle de qualidade do
% produto em granel.
% }
% {Faz. Rio Grande - PR}

\entry{jan/2021}{jun/2020}
{Consultoria técnico-científica}
{MC Química}
{
    Desenvolvimento de produtos de natureza farmacêutica e cosmética, 
    caracterização de geis poliméricos utilizando reologia;
    cromatografia de exclusão de tamanho e medidas de potencial zeta.
}
{Curitiba - PR}

\entry{mar/2020}{aug/2018}
{Pesquisador Terceiro}
{Grupo Boticário}
{
Desenvolvimento de metodologia analítica para predição de estabilidade
de formulações cosméticas utilizando reologia,
caracterização e avaliação da estabilidade acelerada e de prateleira
de formulações cosméticas.
}
{São José dos Pinhais - PR}

\entry{mar/2018}{set/2017}
{Farmacêutico}
{Farmácias Panvel}
{Responsável Técnico}
{Curitiba - PR}

\entry{mar/2016}{jul/2015}
{Assistente de Pesquisa}
{Monash University}
{
Desenvolvimento e simulação computacional da estabilidade 
de estruturas peptídicas cíclicas sintetizadas em laboratório 
com finalidade de produção de nanomateriais com compatibilidade biológica.
}
{Melbourne - Austrália}

\entry{jun/2014}{mar/2012}
{Iniciação Científica}
{Universidade Federal do Paraná}
{
Desenvolvimento de projeto de pesquisa, 
manutenção do laboratório, 
produção e utilização de enzimas para processos biotecnológicos.
}
{Curitiba - PR}