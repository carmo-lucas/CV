\pagestyle{fancy} % tipo de header e footer
\fancyhf{} 
\rfoot{\scriptsize{Updated \today}}
\renewcommand{\headrulewidth}{0pt}

% HEADER
\thispagestyle{empty}
\begin{center}
	{\huge Lucas Chagas Lima do Carmo}
	
	\vspace{6pt}
	
	\begin{tabular}{c|l}
		\arrayrulecolor{Accent}
		+55 41 & 991 216 285
	\end{tabular}
	
	\vspace{6pt}
	
	\textcolor{Accent}{\underline{\href{mailto:msclucaslima@gmail.com}{msclucaslima@gmail.com}}}
	
	\vspace{12pt}
\end{center}


% BODY ----------



\section{Work Experience}

\entry{current}{jun/2020}
{Technical-scientific consultant}
{MC Química}
{Development of pharmaceutical and cosmetic products, 
materials characterization utilizing techniques such as rheology;
size exclusion chromatography and zeta-potential.}

\entry{mar/2020}{aug/2018}
{Researcher}
{Grupo Boticário}
{Development of analytical methods to cosmetic emulsion 
stability prediction utilizing rheology,
characterization and accelerated stability evaluation 
of cosmetic formulas.}

\entry{mar/2018}{sep/2017}
{Pharmacist}
{Panvel Drugstore}
{Dispenser and responsible for training.}

\entry{mar/2016}{jul/2015}
{Research Assistant}
{Monash University}
{Synthesis, purification and computational simulation stability assessment
of cyclical peptides used in biomaterials.}

\entry{jun/2014}{mar/2012}
{Laboratory Assistant}
{Federal University of Paraná}
{Laboratory maintenance, 
utilization of enzymes produced by biotechnological methods
to seek new methods of producing bio fuels.}

\section{Academic Experience}

\entry{mar/2020}{mar/2018}
{Master's in Pharmaceutical Sciences with emphasis in Physico-chemistry}
{Federal University of Paraná}
{The dissertation titled: \emph{Emulsion physical stability prediction by means of rheology}
was written in parthership with Grupo Boticário, where the stability of cosmetic creams was
assessed over time and correlated with physicochemical and rheological parameters to improve 
current methodologies for predicting cosmetic creams stability.}

\entry{sep/2017}{aug/2011}
{Bachelor's degree in Pharmacy}
{Federal University of Paraná}
{Curitiba - Brazil.}

\entry{mar/2016}{mar/2015}
{Bachelor's in Pharmaceutical Sciences}
{Monash University}
{Melbourne - Australia.}

\section{Voluntary work and Internships}

\entry{current}{mar/2020}
{Technical Assistant}
{Federal University of Paraná}
{Production, development and quality control of human sanitizing products
used by COVID-19 relief organizations.}

\entry{jul/2017}{jan/2017}
{Internship}
{University Hospital}
{Internship in clinical analysis in the hematology sectors,
immunochemistry, virology, blood bank and bacteriology.}

\entry{jul/2017}{oct/2016}
{Internship in Teaching}
{Federal University of Paraná}
{Introduction to teaching, university program
in the discipline of Enzymology and fermentation technology.}

\entry{feb/2017}{nov/2016}
{Internship}
{Farmácia Magistral}
{Quality control of raw materials,
formulation of solid dosage forms,
semi-solid and liquid, dispensing formulas,
revision of prescriptions.}

\entry{jan/2017}{mar/2016}
{Internship}
{Panvel Drugstore}
{Dispensing of medicines,
issuing reports to the Health Surveillance,
control of medication storage and
controlled medication storage.}

\section{Skills}

\entry{}{}{English: Advanced/Fluent}{Portugese: Native}

\entry{}{}{Rheology}{Specialist}{I developed rheological characterization 
techniques to predict stability using the latest equipment such as 
TA Instruments DHR-1 and less advanced ones such as Thermo Haake RheoStress RS1.}

\renewcommand{\labelitemi}{\textendash}

\entry{}{}{Analysis Techniques}{Laboratory Equipment}{
	\begin{itemize}
		\item Particle size analysis by optical microscopy and laser diffraction using Malvern Mastersizer 3000;
		\item Differential microcalorimetry analysis;
		\item Analysis of zeta potential for suspensions and emulsions;
		\item Basics of size exclusion chromatography.	
	\end{itemize}}

\entry{}{}{Office Tools}{Microsoft Office, iWork}{Intermediary}

\entry{}{}{Others}{Data Analysis and visualization using the R language.}{}

\end{document}
