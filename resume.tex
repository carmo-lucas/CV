\pagestyle{fancy} % tipo de header e footer
\fancyhf{} 
\rfoot{\scriptsize{Atualizado em \today}}
\renewcommand{\headrulewidth}{0pt}

% HEADER ----

\thispagestyle{empty}

\begin{center}
	
	{\LARGE Lucas C. L. do Carmo, MSc.}
	
	\vspace{6pt}
	\arrayrulecolor{Accent}
	\begin{tabular}{c|c|c}
		(41) 991 216 285 & \textcolor{Accent}{\underline{\href{mailto:msclucaslima@gmail.com}{msclucaslima@gmail.com}}} & Curitiba - PR
	\end{tabular}

	\vspace{6pt}
	
\end{center}
	
% BODY ----------

\section{Experiência}

\entry{atual}{ago/2021}
{Supervisor de Garantia de Qualidade e Controle de Qualidade}
{Cosmetici}
{
	Responsável por supervisionar tratativa de não-conformidades entre setores.
	Monitoramento de processo de fabricação e envase.
	Acompanhamento de não-conformidades e implementação de
	sistema de melhoria contínua e gerenciamento de riscos.
	Atuo na atualização de procedimentos e protocolos de validação dos setores
	de controle de qualidade e de fabricação.
}
{Faz. Rio Grande - PR}

\entry{ago/2021}{mar/2021}
{Analista de Controle de Qualidade Junior}
{Atsum / Cosmetici}
{
	Análise físico-química de matérias-primas e materiais de embalagem.
	Acompanhamento de processos de fabricação, envase e controle de qualidade do
	produto em granel.
}
{Faz. Rio Grande - PR}

\entry{jan/2021}{jun/2020}
{Consultoria técnico-científica}
{MC Química}
{
	Desenvolvimento de produtos de natureza farmacêutica e cosmética, 
	caracterização de matérias-primas utilizando técnicas como reologia;
	cromatografia de exclusão de tamanho e medidas de potencial zeta.
}
{Curitiba - PR}

\entry{mar/2020}{ago/2018}
{Pesquisador Terceirizado}
{Grupo Boticário}
{
	Desenvolvimento de metodologia analítica para predição de estabilidade
	de formulações cosméticas utilizando reologia,
	caracterização e avaliação da estabilidade acelerada e de prateleira
	de formulações cosméticas.
}
{São José dos Pinhais - PR}

\entry{mar/2018}{set/2017}
{Farmacêutico}
{Farmácias Panvel}
{Responsável Técnico.}
{Curitiba - PR}

\entry{mar/2016}{jul/2015}
{Assistente de Pesquisa}
{Monash University}
{
	Desenvolvimento e simulação computacional da estabilidade 
	de estruturas peptídicas cíclicas sintetizadas em laboratório 
	com finalidade de produção de nanomateriais com compatibilidade biológica.
}
{Melbourne - Austrália}

\entry{jun/2014}{mar/2012}
{Iniciação Científica}
{Universidade Federal do Paraná}
{
	Desenvolvimento de projeto de pesquisa, 
	manutenção do laboratório, 
	produção e utilização de enzimas para processos biotecnológicos.
}
{Curitiba - PR}

\section{Formação Acadêmica}

\entry{mar/2020}{mar/2018}
{Mestrado em Ciências Farmacêuticas}
{Universidade Federal do Paraná / Grupo BIOPOL}
{
	A dissertação intitulada: \emph{Emulsion physical stability prediction by means of rheology}
	foi desenvolvida em parceria com o Grupo Boticário onde foi avaliada a estabilidade
	de formulações cosméticas ao longo do tempo e correlacionado com parâmetros 
	físico-químicos para aperfeiçoamento de métodologias para predição de estabilidade de cremes.
}
{Curitiba - PR}

\entry{set/2017}{ago/2011}
{Bacharelado em Farmácia}
{Universidade Federal do Paraná}
{}
{Curitiba - PR}

\entry{mar/2016}{mar/2015}
{Bacharelado em Ciências Farmacêuticas}
{Monash University}
{}
{Melbourne - Austrália}

\section{Trabalho Voluntário e Estágios}

\entry{jan/2021}{mar/2020}
{Assistente Técnico}
{Universidade Federal do Paraná}
{
	Produção, desenvolvimento e controle de qualidade de produtos higienizantes 
	para uso humano para serem usados no combate à COVID-19.
}
{Curitiba - PR}

\entry{jul/2017}{jan/2017}
{Estágio}
{Hospital de Clínicas}
{
	Estágio em análises clínicas nos setores de hematologia, 
	imunoquímica, virologia, banco de sangue e bacteriologia.}
{}
{Curitiba - PR}

\entry{jul/2017}{out/2016}
{Estágio em Docência}
{Universidade Federal do Paraná}
{
	Introdução à docência, programa da universidade 
	na disciplina de Enzimologia e tecnologia de fermentações.
}
{Curitiba - PR}

\entry{fev/2017}{nov/2016}
{Estágio}
{Farmácia Magistral}
{
	Controle de qualidade de matérias primas, 
	formulação de formas farmacêuticas sólidas, 
	semi-sólidas e líquidas, dispensação de fórmulas, 
	revisão de prescrições.
}
{Curitiba - PR}

\entry{jan/2017}{mar/2016}
{Estágio}
{Panvel Farmácias}
{
	Dispensação de medicamentos, 
	emissão de relatórios à Vigilância Sanitária, 
	controle de armazenamento de medicamentos e 
	armazenamento de medicamentos controlados.
}
{Curitiba - PR}

\section{Habilidades e Aptidões}


% \skill
% {Curso}
% {Jornada em Data Science - Escola Ômega - 330 horas}
% {330 horas}
% {
% 	Curso de R aplicado a Ciência de dados com módulos de
% 	estatística, matemática, probabilidade e \emph{Machine Learning}.
% }


\skill
{R}
{3}
{330 horas}
{Aprendi autodidata etc}
	
\skill
{python}
{1}
{120}
{Aprendi autodidata etc}
	




\entry{jan/2021}{mar/2018}{Reologia}{Especialista}
	{
	Desenvolvi técnicas de caracterização reológica para 
	predição de estabilidade utilizando o 
	equipamento de ponta como o TA Instruments DHR-1 e 
	menos avançados como o Thermo Haake RheoStress RS1.
	}{Curitiba - PR}



\entry{}{}{Inglês}{Fluente}{}{}




\renewcommand{\labelitemi}{\textendash}

\entry{}{}{Técnicas de análise}{Equipamentos laboratoriais}
{
	\begin{itemize}
	\item Análise de tamanho de partículas por microscopia óptica e difração laser utilizando Malvern Mastersizer 3000;
	\item Análises de microcalorimetria diferencial;
	\item Análise de potencial zeta para suspensões e emulsões;
	\item Cromatografia por exclusão de tamanho.
	\end{itemize}
	}{}

\entry{}{}{Pacote de Ferramentas}{Microsoft Word, Excel (intermediário)}{}{}

\entry{}{}{Outras}{Análise estatística e visualização de dados utilizando a linguagem R}{}{}
