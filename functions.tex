%Definindo o comando de entrada de itens
\newcommand{\highlight}[1]{
	\textcolor{Accent}{#1}%
}

\newcommand{\entry}[6]{
% 1 - Data final
% 2 - Data inicial
% 3 - Cargo
% 4 - Empresa / Instituição
% 5 - Descrição
% 6 - Local

	\large{#3} \hfill \small{\textcolor{Accent}{#6}} \\
	\textcolor{Accent}{\textbf{\textsc{#4}}} \hfill \textsc{#2} \textcolor{Accent}{-\textrightarrow} \textsc{#1} \\
	\textcolor{Accent}{\vline} \hspace*{0.1cm}
	\begin{minipage}[t]{.75\textwidth}
		\footnotesize{#5}
	\end{minipage}
	\vspace*{5mm}
}

\newcommand{\skill}[3]{
	% 1. Skill
	% 2. Proficiency
	% 3. Description
\begin{minipage}[t]{0.48\textwidth}
	\normalsize 
		\textbf{#1} \hfill
		\color{lightgray}
		\ifthenelse{\equal{#2}{1}}{
			\highlight{\CIRCLE}\CIRCLE\CIRCLE\CIRCLE\CIRCLE 
			}{}
		\ifthenelse{\equal{#2}{2}}{
			\highlight{\CIRCLE\CIRCLE}\CIRCLE\CIRCLE\CIRCLE 
			}{}
		\ifthenelse{\equal{#2}{3}}{
			\highlight{\CIRCLE\CIRCLE\CIRCLE}\CIRCLE\CIRCLE 
			}{}
		\ifthenelse{\equal{#2}{4}}{
			\highlight{\CIRCLE\CIRCLE\CIRCLE\CIRCLE}\CIRCLE 
			}{}
		\ifthenelse{\equal{#2}{5}}{
			\highlight{\CIRCLE\CIRCLE\CIRCLE\CIRCLE\CIRCLE} 
			}{} \\
			\sffamily\color{black}#3
			\vspace*{5mm}
\end{minipage}
}
% Espaçamento e formatação de seções
% Estilo de Section
\titleformat{\section}{\Large\scshape\raggedright}{}{0em}{}%
	[
		\vspace*{-0.667\baselineskip} % Espaço vertical entre texto do título e titlerule.
		\textcolor{Accent} % Cor do texto do título.
		{\titlerule} % Adiciona linha abaixo do título.
	]

\titlespacing{\section}{0pt}{0pt}{0pt}


\newcommand{\icon}[3]{
	\highlight{\faIcon{#1}} \hskip.5em\relax \href{#2}{#3}
}
